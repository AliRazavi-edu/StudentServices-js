% !TeX root = ../Report.tex

\addcontentsline{toc}{section}{مقدمه}
\section*{مقدمه}
در این فصل به مفاهیم و دانش فنی لازم به جهت درک نرم‌افزار پرداخته شده است.

\section{BackEnd}
Backend یا بک اند، به بخشی از یک وب سایت یا نرم افزار می‌گویند که برای کاربران قابل مشاهده نیست. به عبارتی دیگر هسته و مغز یک سایت است که وظیفه کنترل منطق آن را بر عهده دارد. سایت‌های دینامیک به برنامه نویس بک اند نیاز دارند تا منطق سایت را به وسیله زبان‌های برنامه نویسی پیاده‌سازی کنند. کاربران به کدهای نوشته‌شده در بک اند دسترسی ندارند و نمی‌ توانند آن‌ها را مشاهده کنند.
این بخش از سایت مانند قسمتی از کوه یخ است که در زیر سطح آب قرار گرفته است. سمت سرور با بخش سمت کاربر ارتباط مستقیم دارد و به اجزایی که در رابط کاربری طراحی شده‌اند جان می‌بخشد. برنامه نویس بک اند باید اطلاعات را متناسب با اهداف مختلف از پایگاه‌داده دریافت کند و در صورت نیاز پس از پردازش به کاربر نمایش دهد. بنابراین Backend از دو بخش منطق سایت و پایگاه داده تشکیل شده است.

\subsection{JavaScript}
\subsubsection{جاوا اسکریپت چیست؟}
JavaScript که به اختصار JS نیز نامیده می‌شود، یکی از محبوبترین زبان‌های برنامه نویسی است. جاوا اسکریپت زبانی سطح بالا، داینامیک، شی‌گرا و تفسیری است که از شیوه‌های مختلف برنامه نویسی پشتیبانی می‌کند. از این زبان می‌توان برای برنامه نویسی سمت سرور ،
\lr{(Server Side)}
 اپلیکیشن‌های موبایل، بازی و اپلیکیشن‌های دسکتاپ استفاده کرد. بنابراین می‌توان اینگونه برداشت کرد که زبان برنامه نویسی جاوا اسکریپت ، یک زبان همه فن حریف است.
اگر با هر یک از این اصطلاحات آشنایی ندارید نگران نباشید، زیرا در ادامه به توضیح هر یک از آن‌ها خواهیم پرداخت. برای اینکه بهتر متوجه چیستی زبان جاوا اسکریپت شوید، در ابتدا باید جواب سوالاتی مانند زبان کامپایلری چیست و چه تفاوتی با زبان مفسری دارد؟، زبان برنامه نویسی سمت سرور و سمت کاربر به چه نوع زبان‌هایی گفته می‌شود؟ را بدانید. پس از درک این مفاهیم می‌توانید آموزش جاوا اسکریپت را شروع کنید.
همانطور که می‌دانید کامپیوترها تنها به زبان صفر و یک (Binary) صحبت می‌کنند و زبان دیگری را متوجه نمی‌شوند. ما در ابتدا برای برقراری ارتباط با ماشین‌ها سعی کردیم به زبان خود آنها، یعنی زبانی که به زبان صفر و یک نزدیک‌تر است، صحبت کنیم. به این نوع زبان‌ها که به صورت مستقیم با پردازنده در ارتباط‌اند، در اصطلاح، زبان‌های سطح پایین
\lr{(Low Level)}
 گفته می‌شود. از جمله این زبان‌ها می‌توان به اسمبلی اشاره کرد.
اما یادگیری و تسلط به این زبان‌ها برای برنامه نویسان فوق العاده سخت بود. بنابراین متخصصین تصمیم به ساخت زبان‌هایی گرفتند که به زبان انسان‌ها نزدیک‌تر باشد. در اصطلاح به این زبان‌ها، زبان‌های سطح بالا
\lr{(High Level)}
 می‌گویند. زبان‎های سطح بالایی مانند JavaScript کار را برای برنامه‎نویسان ساده‌تر کردند، زیرا ساختار نوشتاری و منطق آن‌ها بسیار به زبان انسان‌ها نزدیک‌تر شده است. پس می‌توان اینگونه نتیجه گرفت که آموزش جاوا اسکریپت نسبت به سایر زبان‌های برنامه نویسی سطح پایین ساده‌تر است.
همانطور که دیدید در تعریف زبان برنامه نویسی جاوا اسکریپت به این نکته اشاره شد که این زبان از نوع زبان‌های مفسری است. برای درک ماهیت زبان‌های برنامه نویسی مفسری ابتدا فکر کنید که شما یک مترجم هستید. برای ترجمه یک متن، دو راه بیشتر ندارید. یا باید آنچه را دریافت می‌کنید به صورت خط به خط و همزمان ترجمه کنید، یا کل مطلب را یک جا ترجمه کنید. این دقیقا همان تفاوت میان زبان‌های مفسری (Interpreter) و زبان‌های کامپایلری (Compiled) است.

\subsubsection{تاریخچه زبان جاوا اسکریپت}
جاوا اسکریپت اولین بار در می‌1995 در 10 روز توسط برندن ایچ، یکی از کارکنان شرکت Netscape متولد شد! در ابتدا این شرکت به این نتیجه رسیده بود که به صفحات وب پویا و جذاب‌تری احتیاج دارد. این اولین قدم به سوی ساخت زبانی ساده بود. آقای براندان ایچ از طرف این شرکت مامور شد که زبانی اسکریپتی برای صفحات وب و دست بردن در کدهای HTML بسازد. ماموریت آقای ایچ این بود زبانی را ارائه کند که نه تنها متخصصان برنامه نویسی از آن استقبال کنند، بلکه به راحتی مورد استفاده طراحان هم باشد.
این شرکت در ابتدا به فکر ارتقا و ساده سازی زبان Schema افتاد اما در نهایت به این نتیجه رسید که به زبانی شبیه جاوا اما با سینتکس ساده‌تر احتیاج دارد. در ابتدای کار اسم این زبان برنامه نویسی Mocha بود که بعد به Mona تغییر پیدا کرد. در سپتامبر همان سال اسم این زبان به LiveScript تغییر کرد و در آخر سریال تغییر اسم با انتخاب اسم JavaScript به اتمام  رسید.
نهایی شدن این اسم تنها به این دلیل بود که در آن روز‌ها زبان برنامه نویسی Java بسیار پرطرفدار شده بود. انتخاب این نام برای این زبان بسیار هوشمندانه بود. زیرا در آن زمان  این زبان با انتخاب این نام، توانست سهم زیادی از بازار جاوا را به خود اختصاص دهد. به هر حال در سال 1996 جاوا اسکریپت برای استاندارد شدن به سازمان ECMA سپرده شد. در نهایت اولین استاندارد جاوا اسکریپت با نام ECMAScript در سال 1997 منتشر شد. اولین اکما اسکریپت ECMA-262 و آخرین ورژن آن با اسم
\lr{ECMAScript 201}7
در ژوئن 2017 منتشر شد.

\subsubsection{نقاط قوت زبان جاوا اسکریپت}
هر یک از زبان هایی که در دنیای برنامه نویسی مورد استفاده قرار می‌گیرند نقاط قوت و ضعف هایی دارند که زبان جاوا اسکریپت هم از این موضوع مستثنا نیست. این زبان به دلیل مزایای فراوانی که دارد در میان برنامه نویسان از محبوبیت زیادی برخوردار است که به طور خلاصه به برخی از آنها اشاره می‌کنیم :
\begin{itemize}
	\item
بر اساس بررسی سایت stackoverflow محبوب‌ترین زبان برنامه نویسی سال 2018 است
\item
برای پردازش و اجرا به کامپایلر احتیاجی ندارد.
\item
یادگیری جاوا اسکریپت نسبت به خیلی از زبان‌های برنامه نویسی راحت‌تر است.
\item
به صورت کراس پلتفورم روی مرورگر‌ها یا پلتفرم‌های مختلف اجرا می‌شود.
\item
نسبت به زبان‌های برنامه نویسی دیگر سبک‌تر و سریع‌تر است.
فریم ورک ها،کتابخانه‌ها و به صورت کلی ابزارهای بسیار زیادی را در اختیارتان قرار می‌دهد.
\item
زبان بومی مرورگر وب است و در مرورگر کاربران پردازش می‌شود.
امکان ایجاد صفحات وب تعاملی و پویا را به برنامه نویسان می‌دهد.
\item
در جواب عمل کاربران، عکس العمل نشان می‌دهد.

\end{itemize}

\subsubsection{نقاط ضعف زبان جاوا اسکریپت چیست؟}
 برخی از ضعف‌های این زبان برنامه نویسی عبارتند از :
\begin{itemize}
	\item
دشواری در تشخیص دلیل خطا دادن و مشکل در دیباگ کردن
\item
محدودیت در اجرای اسکریپت‌های جاوا اسکریپت با ایجاد محدودیت هایی جهت حفظ امنیت
\item
اجرا نشدن بر روی مرورگرهای قدیمی
\item
نفوذپذیری نسبت به اکسپلویت‌ها و عوامل مخرب
\item
می تواند برای اجرای کدهای مخرب در کامپیوتر کاربران استفاده شود.
\item
با رندر شدن متفاوت بر روی ابزارهای مختلف می‌تواند باعث ایجاد تناقض و نداشتن یکپارچگی شود.
\end{itemize}

\subsubsection{کاربرد جاوا اسکریپت}
پیش‌تر به محبوبیت زبان جاوا اسکریپت اشاره کردیم. این محبوبیت بی دلیل نیست چرا که با این زبانِ شی گرا شما قادر خواهید بود تا سایت‌های بی‌روح خود را جان بخشی کنید و با کاربران خود تعامل داشته باشید. یعنی می‌توانید فایل‌های انیمیشنی، صوتی و تصویری را روی سایت خود به نمایش بگذارید. همچنین می‌توانید روی سایت‌تان تایمر قرار دهید، رنگ‌ها را با حرکت موس تغییر دهید و بسیاری کارهای دیگر که باعث جذابیت بیشتر صفحات وب می‌شوند.
اما این تمام چیزی نیست که جاوا اسکریپت در اختیار شما قرار می‌دهد. شما با استفاده از این زبان می‌توانید شروع به ساخت برنامه‌های وب و موبایل و دسکتاپ کنید. برای این منظور می‌توانید از فریم‌ورک‌های مختلف JavaScript که مجموعه‌ای از کتابخانه‌ها را در اختیار شما قرار می‌دهند استفاده کنید. یکی از کارهای سرگرم کننده دیگری که می‌توانید از طریق این زبان انجام دهید، توسعه بازی‌های رایانه ای تحت مرورگر است.  پس به صورت کلی می‌توان کاربردهای زبان جاوا اسکریپت را به صورت زیر بیان کرد :
\begin{itemize}
	\item
برنامه نویسی فرانت اند
\item
برنامه نویسی بک اند با جاوا اسکریپت
\item
برنامه نویسی نرم افزارهای موبایل
\item
برنامه نویسی نرم افزارهای دسکتاپ

\end{itemize}

\subsection{NodeJS}
\subsubsection{Node.js چیست ؟}
Node.js یک پلتفرم سمت سرور مبتنی بر موتور جاوا اسکریپت گوگل کروم
\lr{(V8 engine)}
 می‌باشد. Node.JS تمام چیزهایی که برای اجرای یک برنامه نوشته شده به زبان جاوا اسکریپت را نیاز دارید برایتان فراهم می‌کند. آقای
\lr{Ryan Dahl}
  در سال
\lr{2009 Node.JS}
 را معرفی کرد تا نشان دهد جاوا اسکریپت قدرتمند‌تر از این حرف‌ها است که فقط برای پویاسازی صفحات وب در فرانت اند استفاده شود. در واقع به کمک Node.js زبان برنامه نویسی جاوا اسکریپ به جای اجرا درمرورگر در محیط سرور اجرا می‌شود. Node.js به شما اجازه می‌دهد به آسانی و سادگی برنامه‌های تحت شبکه مقیاس پذیر و بزرگ بنویسید.
جاوا اسکریپت از سال 1995 در حال پیشرفت بود. هر چند این زبان تا مدت‌ها قبل حضور موفقی در سمت سرور نداشت و و تلاش هایی که توسط برنامه نویسان انجام شده بود، به مرور زمان از ذهن توسعه دهندگان دیگر محو می‌شد. تا اینکه با معرفی نود جی اس در سال 2009 مهره برگشت و به مرور زمان جاوا اسکریپت بیشتر و بیشتر در سمت سرور مورد استفاده قرار گرفت.

\subsubsection{چرا باید از Node.js استفاده کنیم؟}
\paragraph{Node.js بازدهی و انعطاف بالایی دارد}
نود در کنار
\lr{V8 engine}
 از زبان برنامه نویسی C++ استفاده کرده و سرعت بسیار بالایی دارد. هم V8 هم Node.js به صورت مرتب آپدیت شده و با قابلیت‌های جدید جاوا اسکریپت هماهنگ می‌شوند، همینطور بازدهی آنها بالاتر رفته و مشکلات امنیتی آنها نیز برطرف می‌شود. همینطور به دلیل استفاده از زبان جاوا اسکریپت انتقال فایل JSON (متداول‌ترین قالب انتقال داده در وب) به طور پیش فرض بسیار سریع خواهد بود.

\paragraph{Node.js کراس پلتفرم است}
پلتفرم هایی مثل Electron.js یا NW.js به شما اجازه می‌دهند با نود جی اس برنامه‌های دسکتاپ بسازید. به این ترتیب می‌توانید برخی از کدهای برنامه تحت وب خود را در محیط ویندوز، لینوکس و مک اواس استفاده کنید. در واقع به کمک نود جی اس، همان تیمی که روی نسخه وب محصول کار می‌کنند، بدون نیاز به دانش تخصصی در زبان‌های
\lr{C\#}
 یا
\lr{Objective C}
  یا سایر زبان هایی که برای ساخت برنامه‌های Native به کار می‌روند، می‌توانند یک برنامه دسکتاپ بسازند.

\paragraph{Node.js می‌تواند با میکروسرویس‌ها ترکیب شود}
اکثر پروژه‌های بزرگ در اول کار ساده بودند و در یک نسخه MVP معرفی شده بودند. اما به مرور زمان این سرویس‌ها بزرگتر شده و نیاز به اضافه کردن قابلیت‌های جدید در آنها حس می‌شد. گاهی وقت‌ها بزرگ شدن سرویس و اضافه کردن امکانات جدید به محصول می‌تواند برای تیم توسعه دهندگان تبدیل به یک کابوس شود. اما یک راه حل مناسب برای حل این مشکل استفاده از میکروسرویس است. میکروسرویس کمک می‌کند برنامه خود را بخش‌های کوچک تقسیم کنید که هر بخش می‌تواند توسط تیم متفاوت و حتی زبانی متفاوت نوشته شود. نود جی اس در کار با میکروسرویس‌ها عملکرد بسیار خوبی دارد.


\subsubsection{Node.js چه کاربردهایی دارد؟}
\paragraph{ساخت برنامه‌های تک صفحه ای (SPA)}
SPA مخفف
\lr{single-page app}
بوده و برنامه هایی گفته می‌شود که تمام بخش‌های آن در یک صفحه پیاده سازی می‌شود. از SPA بیشتر برای ساخت شبکه‌های اجتماعی، سرویس‌های ایمیل، سایت‌های اشتراک ویدئو و غیره استفاده می‌شود. یکی از معروف‌ترین سایت هایی که به این شکل ساخته شده است، سرویس اشتراک ویدئو یوتیوب است. از آنجایی که نود جی اس از برنامه نویسی نامتقارن یا asynchronous به خوبی پشتیبانی می‌کند، برای ساخت برنامه‌های SPA انتخاب خوبی به حساب می‌آید

\paragraph{ساخت برنامه‌های RTA}
RTA مخفف
\lr{real-time app}
 می‌باشد. یعنی برنامه هایی که به صورت لحظه ای دارای تغییرات مختلفی هستند. به احتمال زیاد قبلا با این نوع برنامه‌ها کار کرده اید. برای مثال Google Sheets، Spreadsheets یا Slack از این دست برنامه‌ها هستند. در کل برنامه‌های تعاملی، ابزارهای مدیریت پروژه، کنفرانس‌های ویدئویی و صوتی و سایر برنامه‌های RTA عملیات‌های سنگین ورودی/خروجی انجام می‌دهند.

\paragraph{ساخت بازی‌های آنلاین تحت مرورگر وب}
ایده ساخت چت روم جذاب است، اما جذابیت آن زمانی بیشتر می‌شود که یک بازی هم برای مرورگر وب بنویسید و کنار آن بازی یک چت روم هم ارائه کنید. به کمک نود جی اس می‌توان به توسعه بازی تحت وب پرداخت. در واقع با ترکیب تکنولوژی‌های HTML5 و ابزارهای جاوا اسکریپت ( مثل Express.js یا Socket.io یا غیره ) می‌توانید بازی‌های دوبعدی جذابی مثل
\lr{Ancient Beast}
 یا PaintWar بسازید.

\subsection{NoSQL}
\subsubsection{NoSQL چیست؟}
در برنامه نویسی سنتی، پایگاه‌های داده معمولا از نوع SQL هستند؛ که یک پایگاه داده رابطه ای یا Relational است. پایگاه‌های داده رابطه ای ساده هستند و کار کردن با آن‌ها معمولا بی دردسر و راحت است. اما این نوع از پایگاه‌های داده یک مشکل بزرگ دارند. این مشکل زمانی خود را نشان داد که غول‌های نرم افزاری دنیا مثل گوگل، آمازون و فیسبوک احتیاج به تحلیلِ داده‌های با حجم و تعداد بالا یا همان
\lr{Big Data}
 پیدا کردند.
پایگاه‌های داده رابطه ای به دلیل نوع ساختار خود، برای تحلیل داده‌های بزرگ غیر بهینه، ناکارا و همینطور کند بودند. البته در بعضی موارد هم استفاده از ساختار جدولی که در پایگاه‌های داده رابطه ای استفاده می‌شود تقریبا ناممکن بود. به همین دلیل ذخیره سازی حجم زیادی از داده‌های بی ساختار
\lr{(Non-structured Data)}
 سرعت و کارایی این پایگاه‌های داده را به شدت کاهش می‌داد. تا اینکه پایگاه‌های داده NoSQL پا به عرصه گذاشتند. پس همانطور که حدس می‌زنید، هدف اصلی ایجاد پایگاه‌های داده NoSQL کار با داده‌های بی ساختار و حجیم است.
گفتیم که مشکل پایگاه‌های داده مبتنی بر SQL از نوع ساختار آن‌ها ناشی می‌شود. اما این ساختار چگونه است و چرا چنین مشکلاتی را ایجاد می‌کند؟ برای فهمیدن پاسخ این سوال احتیاج داریم کمی با ساختار پایگاه‌های داده SQL آشنا شویم.

\subsubsection{پایگاه‌های داده NoSQL}
پایگاه‌های داده
\lr{NoSQL (Not Only SQL)}
  برعکس نوع SQL از ساختارهای Schema غیر ثابت یا
Dynamic Schema
 استفاده می‌کنند. این باعث می‌شود که برنامه نویسان احتیاجی به تشکیل ساختارهای سخت گیرانه مشخص، پیش از ایجاد پایگاه‌های داده را نداشته باشند. این پایگاه‌های داده می‌توانند انواع مختلفی داشته باشند و برعکس SQL برای ذخیره سازی داده‌ها از XML یا JSON استفاده می‌کنند. در ادامه انواع مختلفی از پایگاه‌های داده NoSQL را به شما معرفی می‌کنیم:
\begin{itemize}
	\item
پایگاه‌های داده کلید-مقدار یا
\lr{Key-Value Database}
: در این نوع از پایگاه داده اطلاعات در قالب جفت‌های کلید-مقدار یا Key-Value ذخیره می‌شود. کلیدها نقش شناسه هر داده را بازی می‌کند. یعنی می‌توانیم با استفاده از آن‌ها مقادیر مختلف داده را ذخیره یا پیدا کنیم. پایگاه‌های داده کلید-مقدار به دلیل ساده بودن در کارکرد، پرکاربرد‌ترین نوع پایگاه‌های داده NoSQL هستند.
\item
پایگاه‌های داده ستونی یا
\lr{Wide-Column Database}
: شاید تصور کنید پایگاه‌های داده ستونی همان پایگاه‌های داده رابطه ای هستند. اما این فقط ظاهر این گونه پایگاه‌های داده است که شبیه به نوع رابطه ای است. گفتیم که در پایگاه‌های داده رابطه ای لازم است که تعداد و نوع ویژگی‌های هر موجودیت و مقادیر داخل آن مشخص و ثابت باشد. این در حالی است که در پایگاه‌های داده ستونی، هر ستون در رکوردهای مختلف می‌تواند شامل داده هایی با ساختار و نوع متفاوت باشد.
\item
پایگاه‌های داده سندی یا
\lr{Document Database}
: در این گونه پایگاه‌های داده برای ذخیره سازی داده‌ها از اسناد JSON یا XML استفاده می‌کنیم. پایگاه‌های داده سندی معمولا برای ذخیره سازی و استفاده از داده‌های پراکنده و بی ساختار استفاده می‌شوند.
\item
پایگاه‌های داده گرافی یا
\lr{Graph Database}
: در این نوع از پایگاه‌های داده برای ذخیره سازی موجودیت‌ها و روابط بین آن‌ها از گراف استفاده می‌کنیم. پایگاه‌های داده گرافی برای مواردی که در آن‌ها به ایجاد ارتباط‌های متعدد بین جداول احتیاج داریم بسیار مناسب هستند.
\item
پایگاه‌های داده چند مدله یا
\lr{Multimodel Database}
: پایگاه‌های داده چند مدله ترکیبی از انواع دیگر پایگاه داده هستند. در این نوع پایگاه‌های داده می‌توانیم داده‌ها را به روش‌های مختلفی ذخیره، و از آن‌ها استفاده کنیم.

\end{itemize}

\subsubsection{مزیت‌های استفاده از NoSQL}
پایگاه‌های داده NoSQL مزیت‌های بسیار زیادی دارند که آن‌ها را برای سیستم‌های بزرگ و توزیع شده تبدیل به بهترین گزینه می‌کند. به طور کلی می‌توان این مزیت‌ها را به این شکل خلاصه کرد:
\begin{itemize}
	\item
مقیاس پذیری بالا (Scalability): پایگاه‌های داده NoSQL می‌توانند به راحتی با روش مقیاس پذیری افقی یا Horizontal Scaling گسترش پیدا کنند. این ویژگی باعث کم شدن پیچیدگی و هزینه مقیاس دادن به نرم افزار یا Scale کردن آن می‌شود.
\item
کارایی بالا (Performance): در سیستم‌های توزیع شده NoSQL با تکثیر خودکار داده‌های NoSQL در سرورهای متعدد در سراسر دنیا، تاخیر در ارسال پاسخ از طرف سرور به پایین‌ترین حد ممکن می‌رسد.
\item
دسترسی بالا (Availability): در سیستم‌های توزیع شده NoSQL به دلیل کپی شدن خودکار داده‌ها در سرورهای مختلف، با از دسترس خارج شدن یک یا چند سرور، پایگاه داده همچنان قابل دسترس و پاسخگو است.
\end{itemize}

\subsection{MongoDB}
\subsubsection{mongo db چیست؟}
مونگو دیبی
\lr{(Mongo DB)}
 یکی از معروف‌ترین پایگاه داده‌های No SQL است که ساختار منعطفی دارد و بیشتر در پروژه هایی با حجم بالای داده استفاده می‌شود. این پایگاه داده پلتفرمی متن باز و رایگان است و با مدل داده‌های مستند گرا
\lr{(Document - Oriented)}
 کار می‌کند و در ویندوز، مکینتاش و لینوکس قابل استفاده است. مقادیر داده ای ذخیره شده در مونگو دیبی، با دو کلید اولیه
\lr{(Primary Key)}
 و ثانویه
\lr{(Secondary Key)}
  مورد استفاده قرار می‌گیرند.
مونگو دیبی شامل مجموعه ای از مقادیر است. این مقادیر به صورت سندهایی (Document) هستند که با اندازه‌های مختلف، انواع مختلفی از داده‌ها را در خود جای داده اند. این مسئله باعث شده که مونگو دیبی بتواند داده هایی با ساختار پیچیده مانند داده‌های سلسله مراتبی و یا آرایه ای را در خود ذخیره کند.

\subsubsection{ویژگی‌های mongo db}
\begin{itemize}
	\item
مونگو دیبی به علت مستند گرا بودن مدل ذخیره داده‌ها در مقایسه با دیتابیس‌های رابطه ای بسیار منعطف‌تر و مقیاس پذیر‌تر است و بسیاری از نیازمندی‌های کسب و کارها را برطرف می‌کند.
\item
این پایگاه داده برای تقسیم داده‌ها و مدیریت بهتر سیستم از شاردینگ (Sharding) استفاده می‌کند. شاردینگ به معنی تکه تکه کردن است و در لود بالای شبکه انجام می‌شود. به گونه ای که دیتابیس به چند زیربخش تقسیم می‌شود تا روند پاسخ دهی به درخواست هایی که از سمت سرور می‌آید، راحت‌تر شود.
\item
داده‌ها با دو کلید اولیه و ثانویه قابل دسترسی هستند و هر فیلدی قابلیت کلید شدن را دارد. این امر زمان دسترسی و پردازش داده را بسیار سریع می‌کند.
\item
همانند سازی (Replication ) یکی دیگر از خصوصیات مهم مونگو دیبی است. در این تکنیک از یک داده به عنوان داده اصلی کپی هایی تهیه شده و بخش‌های دیگری از سیستم پایگاه داده ذخیره می‌شود. در صورت از بین رفتن و یا مخدوش شدن این داده، داده‌های کپی شده به عنوان داده اصلی و جایگزین مورد استفاده قرار می‌گیرند.
\end{itemize}

\subsubsection{روش کار mongo db}
در دیتابیس‌های رابطه ای داده‌ها به شکل رکورد (Record) نگهداری می‌شوند اما در مونگو دیبی، ساختار نگهداری داده‌ها به شکل سند است. هر سند از نوع
\lr{Binary JSON}
 یا BSON است و دارای فیلدهای کلید و مقدار می‌باشد.
برای اجرا کردن کدهایی که در مونگو دیبی نوشته شده است باید از طریق
\lr{Mongo Shell}
 اقدام کرد. مونگو شل رابط تعاملی دیتابیس و برنامه نویس محسوب می‌شود و به آن‌ها اجازه ارسال کوئری (Query) و به روزرسانی داده‌ها را می‌دهد.

\subsubsection{مزایا و معایب mongo db}
دیتابیس‌های رابطه ای دارای اسکیما (Schema) هستند. یعنی ساختار خاصی برای داده هادر نظر گرفته و مدل‌های محدودی را ذخیره می‌کنند. اما مونگو دیبی و به طور کلی دیتابیس‌های NoSQL در برابر پذیرش داده هایی با توع مختلف بسیار منعطف هستند و این مزیت مهمی برای برنامه نویسان محسوب می‌شود. مقیاس پذیری این پایگاه داده باعث استفاده از آن در پروژه هایی می‌شود که با کلان داده‌ها
\lr{(Big Data)}
 سروکار دارند.
علاوه بر مزایای گفته شده مشکلاتی نیز در مونگو دیبی وجود دارد که ممکن است دردسرساز شود. این دیتابیس در استفاده از کلید خارجی
\lr{(Foreign Key)}
 برای داده‌ها ضعف دارد و ممکن است پایداری داده‌ها و یکپارچگی سیستم را به هم بریزد. همچنین در خوشه بندی داده‌های موجود در این پایگاه داده، تنها می‌توان یک گره (Node) را به عنوان گره اصلی (Master) انتخاب کرد که اگر از بین برود، ممکن است مرتب سازی زیرگره‌های آن از بین برود. این مشکل در پایگاه داده کاساندرا (Cassandra) برطرف شده است.


\section{FrontEnd}
فرانت اند یا
\lr{Front End}
، به بخش قابل مشاهده‌ی یک وب سایت یا نرم افزار توسط کاربران می‌گویند. فرانت اند، کدهای غیر قابل فهم برای کاربران را در قالب ظاهری گرافیکی و بصری به آن‌ها نمایش می‌دهد تا بتوانند به راحتی از بخش‌های مختلف سایت استفاده کنند. در این بخش، فرم‌های ورودی اطلاعات، صداها، تصاویر، ویدئوها و به صورت کلی هر چیز دیگری که برای کاربر قابل درک باشد، قرار می‌گیرد.
فرانت اند به دو بخش اصلی طراحی و توسعه رابط کاربری تقسیم می‌شود. در بخش طراحی، طراحان با نرم افزارهای گرافیکی مانند فتوشاپ، ادوبی ایکس دی، فیگما و... ظاهر سایت را طراحی می‌کنند. اما بخش توسعه‌ی رابط کاربری مربوط به پیاده سازی ظاهر سایت در قالب کدهای
\lr{HTML ،CSS, JS}
است. بخش قابل مشاهده‌ سایت برای کاربران در سمت فرانت را سمت کاربر یا
\lr{Client Side}
 می‌نامند. بنابراین کدهای نوشته شده در سمت فرانت اند، در مرورگر کاربر پردازش و اجرا می‌شوند. یعنی کاربر به راحتی به این کدها دسترسی مستقیم دارد و می‌تواند آن‌ها را مشاهده کند. فرانت اند 
 \lr{(Front-end)}
  با بخش بک اند 
  \lr{(Back-end)}
   در ارتباط مستقیم است و بر روی تجربه کاربران هنگام استفاده از محصول تاثیر بسیاری می‌گذارد.

\subsection{Html}
\subsubsection{HTML چیست؟}
HTML
 مخفف 
\lr{Hyper Text Markup Language}
 بوده و در فارسی به آن زبان نشانه‌ گذاری ابرمتن می‌گویند. دقت کنید که HTML یک زبان برنامه نویسی نیست، بلکه یک زبان نشان‌گذاری یا 
\lr{Markup language}
  به حساب می‌آید.

\subsubsection{تاریخچه زبان HTML}
برای اینکه بدانیم HTML از کجا آمده باید سفر کوتاهی به سال 1991 داشته باشیم. زمانی که آقای 
\lr{Tim Berners-Lee}
 کار خود را روی 
\lr{18 Tag}
  یا همان برچسب ساده شروع کرد و اولین نسخه HTML را طراحی کرد. HTML روز به روز پیشرفت کرد و در هر نسخه امکانات بیشتری را در قالب تگ‌‌های کاربردی‌تر در اختیار طراحان قرار داد.

به این ترتیب این زبان مشکلات قبلی خود را به مرور رفع کرد. HTML4 در سال 1999 معرفی شد و توانست تا مدت‌ها توسط طراحان وب مورد استفاده قرار گیرد، تا این که بزرگترین تحول تاریخ HTML با معرفی HTML5 اتفاق افتاد. این نسخه از زبان HTML توانست بیش از پیش به توسعه دهندگان در طراحی سایت‌ها کمک کند که در ادامه می‌خواهیم با آن بیشتر آشنا شویم.


\subsubsection{HTML چطور کار می‌کند؟}
HTML عناصر مختلفی را از جمله پاراگراف، لیست، عکس، صوت و غیره کنار هم قرار می‌دهد تا چهارچوب اصلی صفحه وب را ایجاد کند. به زبان ساده‌تر ما با HTML بدنه اصلی صفحه وب را می‌سازیم.
اگر HTML را شبیه به یک ساختمان در حال ساخت در نظر بگیریم، مهندس عمران که پی ساختمان را ریخته و اسکلت آن را می‌سازد حکم کسی را دارد که ساختار اصلی صفحات وب را با HTML می‌سازد. همچنین مهندس معماری که وظیفه دارد ظاهر ساختمان را زیباتر کند مانند کسی است که به کدنویسی با CSS می‌پردازد.
البته در دنیای وب معمولا وظیفه کدنویسی HTML و CSS به عهده یک نفر خواهد بود. فایل‌‌های HTML با پسوند .htm یا .html در سیستم ذخیره می‌شوند. این فایل‌ها تقریبا توسط همه مرورگرهای وب پشتیبانی می‌شوند و به راحتی می‌توانند محتویات آن را رندر کنند. منظور از رندر کردن این است که عناصر داخل سایت که ترکیبی از کد، تصویر، انیمیشن، ویدئو یا غیره هستند، تبدیل به اطلاعات قابل نمایش برای کاربران می‌شوند.

\subsubsection{تگ چیست؟}
HTML به کمک برچسب‌‌ها (Tags) عناصر مختلف را کنار هم می‌چیند و هر کاربر با توجه به نیاز خود از آن‌ها استفاده می‌کند. شاید بپرسید تگ چیست؟ تگ‌ها عناصری هستند که وظایف گوناگونی دارند و با فراخوانی هر کدام کارشان شروع شده و با بستن تگ کارشان تمام می‌شود. مثلا برای نوشتن پاراگراف‌ها در زبان HTML از تگ p استفاده می‌شود و زمانی که پاراگراف تمام شده، تگ هم بسته می‌شود. همچنین برای نشان دادن لینک‌ها از تگ a استفاده در صفحات وب استفاده می‌شود.
تگ‌های HTML در حقیقت همان دستورالعمل‌های این زبان هستند که به مرورگر می‌گویند صفحه مورد نظر از چه عناصری تشکیل شده است. هر کدام از این Tag معنا و مفهوم خاصی دارند و به شما امکاناتی مانند تغییر شکل ظاهری متن‌ها، ساخت لیست‌های مختلف و به هم متصل کردن صفحات را می‌دهند. همچنین از آن‌ها برای کار با صدا، تصویر و غیره استفاده می‌شود.

\subsubsection{چرا HTML یک زبان برنامه نویسی نیست؟}
HTML
 هرگز نمی‌تواند یک زبان برنامه‌نویسی باشد. زیرا اصلا ویژگی‌های یک زبان برنامه‌نویسی، مثل متغیر‌ها، توابع، شرط‌ها، حلقه‌ها و… را ندارد. پس کاملا اشتباه است اگر HTML را یک زبان برنامه‌نویسی بدانیم. می‌توانیم درباره‌ی HTML بگوییم که ابزاری است که با استفاده از تگ‌ها، می‌تواند صفحات وب را برای ما ساختاردهی کند.

\subsubsection{مزایا و معایب زبان HTML چیست؟}
HTML در کنار css و js هسته اصلی وب را تشکیل می‌دهد و یک زبان بسیار مهم در دنیای وب حساب می‌شود. این زبان مزیت‌ها و محدودیت‌هایی هم دارد که در ادامه به آن‌ها اشاره می‌کنیم و می‌بینیم دلیل اصلی ماندگاری HTML چیست و چرا این زبان با تمام مشکلاتش هنوز زبان شماره یک وب به حساب می‌آید. برخی از مهمترین مزایا و معایب این زبان عبارتند از:

\paragraph{مزایای HTML :}
\begin{itemize}
	\item 
	یادگیری آسان و لذت‌بخش
	\item 
	قابلیت اجرا در تمام مرورگرها
	\item 
	متن باز و رایگان بودن
	\item 
	ادغام آسان با زبان‌های سمت سرور مثل php
\end{itemize}


\paragraph{معایب HTML :}
\begin{itemize}
	\item 
	استاتیک بودن و وابستگی به زبان‌های سمت سرور برای تعامل با کاربر
	\item 
	ضعف در پشتیبانی از مرورگرهای قدیمی
	\item 
	نیاز به طراحی جداگانه هر صفحه به دلیل نبود قواعد منطقی برنامه نویسی
\end{itemize}


\subsection{Css}
\subsubsection{css چیست؟}
سی اس اس مخفف 
\lr{Cascading Style Sheet (CSS)}
 است. زبان css یک زبان طراحی صفحات وب برای ایجاد و ساخت مشخصات ظاهری اسناد و اطلاعات وب سایت می باشد. css یکی از رایج ترین و محبوب ترین ابزارهای طراحی صفحات وب سایت نوشته شده توسط زبان HTML و یا XHTML می باشد و همچنین از زبان های اسکریپت دیگری مانند ، SVG
 \lr{plain XML}
  و XUL نیز به خوبی پشتیبانی می نماید.

در کدنویسی با استفاده از CSS می‌توانید استایل سایت مثل رنگ، فونت، تصاویر پس زمینه و … را بصورت دلخواه تغییر دهید.

\subsubsection{هدف و کاربرد css چیست ؟}
هدف از تولید css در واقع جداسازی اطلاعات محتوا (که توسط زبانی مانند HTML نوشته شده اند) از اطلاعات ظاهری مانند صفحه بندی، رنگ و سایز و نوع فونت می باشد. این جداسازی موجب افزایش سرعت در دسترسی به سایت، انعطاف پذیری بیشتر برای کنترل ویژگی های ظاهری، قابلیت طراحی چندین صفحه با یک فرمت یکسان و جلوگیری از پیچیدگی و انجام کارهای تکراری در طراحی وب سایت می گردد.

از دیگر کاربرد css این است که می توان تنظیماتی را اعمال نمود که نمایش صفحه وب سایت مورد نظر بسته به اندازه صفحه نمایش کاربر متغیر باشد که به آن اصطلاحا طراحی ریسپانسیو می گویند. در صورتی که مدیر وب سایت چندین نوع نمایش را برای یک صفحه وب سایت خود تنظیم نموده باشد،  css برای تصمیم گیری اینکه کدام حالت را به نمایش بگذارد، از ابزارهای تعیین اولویت استفاده می نماید.

\subsubsection{مزایای سی اس اس چیست ؟}
\begin{itemize}
	\item 
	سازگاری بیشتر در طراحی
	\item 
	گزینه های قالب بندی بیشتر
	\item 
	کد سبک
	\item 
	بارگیری سریعتر
	\item 
	بهینه سازی موتور جستجو
	\item 
	دسترسی بهتر به کد
\end{itemize}


\subsubsection{معایب سی اس اس چیست ؟}
\begin{itemize}
	\item 
	کنترل ضعیف صفحه بندی های قابل انعطاف
	\item 
	عدم امکان انتخاب گزینه های والد
	\item 
	محدودیت در کنترل فرم های عمودی
	\item 
	عدم وجود توضیحات لازم در زبان css 
	\item 
	بروز مشکلاتی در ساخت ستون ها
\end{itemize}


\subsection{Scss چیست ؟}
Scss پسوند نحوی CSS است. این بدان معنی است که هر شیوه نامه معتبر CSS یک پرونده SCSS معتبر با همان معنی است. علاوه بر این ، SCSS بیشتر هک های CSS و نحو اختصاصی فروشنده ، مانند فیلتر قدیمی IE را می فهمد.


\subsection{Figma}
\subsubsection{فیگما چیست؟}
فیگما یک برنامه‌ی طراحی رابط کاربری است که روند ساخت آن در سال 2012 شروع شد و نهایتا در سال 2016 اولین نسخه از آن در اختیار عموم قرار گرفت. فیگما در حالت کلی با دیگر ابزارهای ویرایشی متفاوت است؛ زیرا یک ابزار تحت وب 
\lr{(web based)}
 است: یعنی به صورت مستقیم در مرورگر اجرا می‌شود. این به این معنی است که شما می‌توانید در هر زمانی به پروژه‌های خود دسترسی پیدا کنید و بدون نیاز به خرید مجوز و یا نصب نرم‌افزار از هر کامپیوتری و یا پلتفرمی کار با آن را شروع کنید. نیازی نیست نگران حجم مصرفی اینترنت خود باشید، چراکه فیگما مصرف پهنای باند زیادی ندارد. با این حال، کاربران ویندوز و مک می‌توانند به راحتی فیگما را روی سیستم عامل خود نصب کرده و از برخی از امکانات آن‌ به صورت آفلاین استفاده کنند.

فیگما یک برنامه‌ی رایگان است و در آن شما می‌توانید تا حتی سه پروژه‌ی فعال را به صورت همزمان ایجاد و ذخیره کنید. این سیاست به شما فرصت می‌دهد تا کار با نرم افزار را یاد بگیرید، آزمایش کنید و روی پروژه‌های کوچک مشغول به طراحی شوید. همچنین فارسی نوشتن در فیگما (در محیط ویرایشگر) بدون هیچ مشکلی امکان پذیر است که خبر خوشی برای طراحانی است که نیازمند استفاده از زبان فارسی هستند.

رابط کاربری 
\lr{(User Interface)}
، یک عضو جداناپذیر از عوامل موثر بر بازاریابی و ارتباط کاربران با کسب و کارهای امروزی به شمار می‌رود. یک رابط کاربری خوب در اغلب موارد با نیازهای کاربران سنجیده می‌شود و اگر به درستی طراحی شود، کاربر برای کار با آن نیازی به آموزش ندارد و از کار کردن با آن لذت می‌برد. به همین دلیل نرم افزارهای متعددی برای ساخت یک رابط کاربری مناسب به وجود آمده‌اند.

یکی از برنامه‌های طراحی رابط کاربری، فیگما (Figma) است. با استفاده از فیگما، می‌توان یک رابط کاربری را به راحتی و در زمانی بسیار کم، به گونه‌ای طراحی کرد که به اصطلاح 
\lr{User Friendly}
 باشد.
\subsubsection{ویژگی‌های فیگما}
\begin{itemize}
	\item 
	\textbf{استفاده از کامپوننت (\lr{Components})}
	: ساخت و ترکیب چند شی با یکدیگر (مشابه با نرم افزار \lr{Sketch})
	\item 
	\textbf{ساخت نمونه‌ی اولیه یا پروتوتایپ}
	: قابلیت کلیک کردن روی کامپوننت‌های مختلف در آن (مشابه با نرم افزار \lr{InVision})
	\item 
	\textbf{کامنت گذاری هنگام طراحی}
	: اضافه کردن، نشانه‌گذاری و پاسخ دادن نظرات توسط اعضای تیم در حال کار روی پروژه
	\item 
	\textbf{برنامه نویسی بدون کد}
	: قابلیت دریافت ابعاد و ویژگی‌ها در قالب فایل استایل‌، دانلود آیکون‌ها و عکس‌ها را از URL پروژه (مشابه با نرم افزار \lr{Zeplin})
	\item 
	\textbf{تنظیم فضا‌ها}
	: ساخت کامپوننت‌ با قابلیت تغییر اندازه با استفاده از تنظیمات فضاها (مشابه با نرم افزار \lr{Sketch})
	\item 
	\textbf{مدیریت پروژه با 
		\lr{Version Control}}
	: مشاهده‌ی تاریخچه و بازگشت به یک نسخه‌ی خاص توسط اعضای تیم
	\item 
	\textbf{کار همزمان توسط چند نفر}
	: مشاهده‌ی نشانگر موس سایر اعضای تیم هنگام کار بر صفحه (هرچند به طور کلی این کار توصیه نمی‌شود.)
	\item 
	\textbf{پلاگین‌ها}
	: استفاده از فونت‌ها، استایل‌های سایر کاربران در قالب پلاگین‌های کاربردی
\end{itemize}


\section{الگوی معماری MVC}
\subsection{MVC‌ چیست ؟}
معماری mvc یک نوع استاندارد کد نویسی برای طراحی سایت حرفه ای با زبان های مختلف مثل
\lr{php، asp.net}
 و ... می باشد ، mvc در واقع مخفف کلمه‌های
\lr{model، view، controller}
  می‌باشد.
کلمه mvc مخفف
\lr{model، view، controller}
 می باشد و وظیفه اصلی آن به زبان ساده و مختصر جدا کردن بخش های منطقی برنامه از بخش های سمت کاربر می باشد ، بدین ترتیب تمامی بخش های منطقی از سمت کاربر جدا شده و در نتیجه انجام تغییرات و توسعه دادن یک سایت یا یک سیستم تحت وب برای تیم برنامه نویسی بسیار راحت تر قابل انجام خواهد بود.

\subsection{تاریخچه MVC}
الگوی معماری MVC ابتدا در ‌اواخر دهه ۱۳۵۰ شمسی توسط تریگو رینسکویگ
\lr{(Trygve Reenskaug)}
 مطرح شد. اولین گزارش راجع‌به MVC زمانی نوشته شد که رینسکویگ با دانشمندی در آزمایشگاه علوم و تحقیقات زیراکس
 \lr{(Zerox PARC)}
  در سال ۱۳۵۷ دیدار می‌کرد. در ابتدا، MVC با نام
\lr{«Thing Model View Editor»}
 شناخته می‌شد، اما خیلی زود نام آن به
\lr{«Model View Controller»}
 تغییر داده شد. هدف تریگو این بود که مشکل کاربران در مدیریت یک پایگاه‌داده بزرگ و پیچیده را برطرف کند.
کاربرد MVC‌ در طول سال‌ها تغییر کرده است. به دلیل اینکه MVC‌ قبل از مرورگرهای وب اختراع شده است، در ابتدا به عنوان یک الگوی معماری برای رابط کاربری گرافیکی (GUI) مورد استفاده قرار می‌گرفت. مدل MVC اولین بار در اواسط دهه شصت شمسی در زبان برنامه‌نویسی Smalltalk ارائه شد. سپس، الگوی معماری MVC برای اولین بار به عنوان یک مفهوم عمومی به صورت یک مقاله در سال 1367 پذیرفته شد. پس از مدتی، الگوی معماری MVC به میزان زیادی در اپلیکیشن‌های تحت وب مورد استفاده قرار گرفت.

\subsection{ویژگی های MVC}
\begin{itemize}
	\item
	آزمایش‌پذیری ساده و بدون دردسر؛ یک فریم‌ورک قابل آزمایش، گسترش‌پذیر و قابل شخصی‌سازی به میزان زیاد.
	\item
	امکان مدیریت و کنترل کامل روی HTML و همچنین URLها در الگوی معماری MVC
	\item
	بهره‌مندی از ویژگی‌های موجود در Django ،JSP ،ASP.NET و سایر موارد در MVC
	\item
	تفکیک بسیار واضح منطق وظایف اپلیکیشن به صورت مدل، نما و Controller‌ در الگوی معماری MVC
	\item
	مسیریابی URL‌ برای URLهای سازگار با سئو (SEO) و نگاشت URL قدرتمند برای URLهای قابل درک و قابل جستجو
	\item
	پشتیبانی از توسعه آزمون‌محور
	\lr{(Test Driven Development | TDD)}
\end{itemize}


\subsection{اجزای MVC}
الگوی معماری MVC در بخش‌های قبلی مطلب «\lr{MVC} چیست» به طور ساده بیان شد، اکنون در این بخش، سه لایه مهم الگوی معماری MVC با جزئیات بیش‌تری مورد بررسی قرار خواهند گرفت. همان‌طور که بیان شد، سه جزء الگوی معماری MVC به شرح زیر است.
\begin{itemize}
	\item
	مدل (Model): این لایه شامل تمام داده‌ها و منطق مرتبط با آن داده‌ها است.
	\item
	کنترل‌گر (Controller): یک رابط میان قطعات مدل و نما به حساب می‌آید.
	\item
	نما (View): داده‌ها را به کاربر نمایش و ارائه می‌دهد یا در واقع تعاملات کاربر را مدیریت می‌کند.
\end{itemize}

در ادامه، هر یک از این قطعات و اجزاء الگوی معماری MVC به طور جداگانه شرح داده شده است.

\subsubsection{مدل (Model)}
مدل به عنوان پایین‌ترین سطح در ساختار معماری برنامه کاربردی شناخته می‌شود. این مسئله، به این معنا است که لایه مدل مسئولیت نگهداری و مدیریت منطقی داده‌ها را بر عهده دارد. مدل در واقع به پایگاه‌داده متصل است. بنابراین، هر کاری که با داده‌ها انجام می‌شود، از قبیل افزودن یا دریافت داده‌ها در بخش مدل انجام می‌شود. قطعه یا لایه مدل، داده‌ها و منطق مربوط به آن‌ها را ذخیره‌سازی می‌کند. مدل به درخواست‌های Controller‌ پاسخ می‌دهد. زیرا Controller‌ هیچ‌گاه به طور مستقیم و به خودی خود با پایگاه‌داده در ارتباط نیست. مدل با پایگاه‌داده مکاتبه می‌کند و سپس داده‌های مورد نیاز را به Controller انتقال می‌دهد. همچنین، مدل هیچ وقت به صورت مستقیم با لایه نما ارتباط ندارد.

\subsubsection{کنترل‌گر (Controller)}
Controller نقش اصلی را ایفا می‌کند. زیرا Controller بخشی است که امکان ارتباط دو طرفه میان مدل و نما را به وجود می‌آورد. بنابراین، Controller نقش یک میانجی را دارد. Controller نیازی به دخالت در مدیریت منطق داده‌ها ندارد و فقط آنچه باید انجام شود را به مدل انتقال می‌دهد. پس از دریافت داده‌ها از مدل، Controller‌ آن‌ها را پردازش می‌کند و سپس تمام آن اطلاعات را برداشته و به نما ارسال می‌کند و نحوه نمایش آن‌ها را به نما توضیح می‌دهد. باید توجه داشت که نما و مدل مستقیماً امکان مکاتبه و ارتباط ندارند.

\subsubsection{نما (View)}
بازنمایی داده به وسیله قطعه یا لایه نما (View) انجام می‌شود. نما در واقع برای کاربر یک رابط کاربری یا UI تولید می‌کند. بنابراین، در خصوص کاربردهای وب، می‌توان قطعه نما را همان بخش HTML و CSS در نظر گرفت. نماها به وسیله داده‌ها ساخته می‌شوند و داده‌ها توسط قطعه مدل گردآوری می‌شوند اما، این داده‌ها به صورت مستقیم از مدل به نما انتقال داده نمی‌شوند، بلکه این کار از طریق Controller صورت می‌گیرد

\subsection{مزایای MVC}
در این بخش از مطلب فواید اساسی استفاده از الگوی معماری MVC فهرست شده است.
\begin{itemize}
	\item
	نگهداری کد
	\lr{(Code maintenance)}
	، تعمیم و رشد آن در یک چارچوب مبتنی بر الگوی معماری MVC بسیار ساده‌تر است.
	\item
	قطعات در الگوی MVC‌ قابل استفاده مجدد هستند.
	\item
	یک قطعه مبتنی بر الگوی معماری MVC‌ را می‌توان به طور مستقل از کاربر آزمایش کرد.
	\item
	پشتیبانی آسان برای مشتریان جدید، به واسطه الگوی معماری MVC امکان‌پذیر می‌شود.
	\item
	در الگوی معماری MVC توسعه قطعات مختلف را می‌توان به صورت موازی انجام داد.
	\item
	الگوی معماری MVC با تقسیم‌بندی یک اپلیکیشن به سه واحد مستقل نما، مدل و Controller از بروز پیچیدگی جلوگیری می‌کند.
	\item
	الگوی معماری MVC تنها از یک الگوی
	\lr{Front Controller}
	 استفاده می‌کند. این الگو، درخواست‌های ایجاد شده در وب‌اپلیکیشن را تنها از طریق یک Controller پردازش می‌کند.
	\item
	الگوی معماری MVC بهترین پشتیبانی را برای توسعه آزمون‌محور
	\lr{(Test-Driven Development)}
	 ارائه می‌دهد.
	\item
	الگوی MVC با وب‌اپلیکیشن‌های پشتیبانی شده توسط گروه‌های بزرگ طراحان و توسعه‌دهندگان، به خوبی کار می‌کند.
	\item
	تفکیک دغدغه‌ها
	\lr{(Clean Separation of Concerns |‌ SoC)}
	 در MVC واضح و شفاف است.
	\item
	الگوی معماری MVC با بهینه‌سازی موتور جستجو (سئو | SEO) سازگاری دارد.
	\item
	تمام کلاس‌ها و اشیاء مستقل از یکدیگر هستند تا بتوان هر یک را به طور مجزا آزمایش کرد.
	\item
	الگوی معماری MVC امکان گروه‌بندی منطقی اعمال مرتبط با یکدیگر در یک Controller را فراهم می‌کند.
\end{itemize}

\subsection{معایب MVC}
در ادامه این بخش از مطلب، معایب MVC و نقاط ضعف آن فهرست شده است.
\begin{itemize}
	\item
	پیچیدگی MVC بالاست.
	\item
	برای اپلیکیشن‌های کوچک کارایی ندارد و مناسب نیست.
	دسترسی به داده‌ها در لایه نما کارایی لازم را ندارد.
	\item
	خواندن، تغییر دادن، آزمایش واحدها و استفاده مجدد از یک برنامه توسعه داده شده مبتنی بر الگوی معماری MVC دشوار است.
	\item
	ناوبری فریم‌ورک MVC ممکن است گاهی با اضافه شدن لایه‌های انتزاعی جدید پیچیده شود. این مسئله نیازمند این است که کاربران با معیارهای تجزیه MVC تطبیق دهند.
	\item
	پشتیبانی رسمی از تایید اعتبار در MVC وجود ندارد.
	\item
	پیچیدگی و عدم کارایی داده‌ها در MVC
	\item
	استفاده از الگوی معماری MVC با رابط کاربری امروزی دشوار است.
	\item
	برای اجرای برنامه‌نویسی موازی به چندین برنامه‌نویس نیاز است.
	\item
	در الگوی معماری MVC نیاز به دانش در چندین حوزه فناوری وجود دارد.
	\item
	نگهداری از حجم بالای کدها در لایه Controller چالش ایجاد می‌کند.
\end{itemize}



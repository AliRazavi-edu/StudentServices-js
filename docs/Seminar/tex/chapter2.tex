\chapter{مفاهیم عمومی، دانش فنی}
\addcontentsline{toc}{section}{مقدمه}
\section*{مقدمه}
\section{JavaScript}
\subsection{جاوا اسکریپت چیست؟}
JavaScript که به اختصار JS نیز نامیده می‌شود، یکی از محبوبترین زبان‌های برنامه نویسی است. جاوا اسکریپت زبانی سطح بالا، داینامیک، شی‌گرا و تفسیری است که از شیوه‌های مختلف برنامه نویسی پشتیبانی می‌کند. از این زبان می‌توان برای برنامه نویسی سمت سرور (Server Side)، اپلیکیشن‌های موبایل، بازی و اپلیکیشن‌های دسکتاپ استفاده کرد. بنابراین می‌توان اینگونه برداشت کرد که زبان برنامه نویسی جاوا اسکریپت ، یک زبان همه فن حریف است.

اگر با هر یک از این اصطلاحات آشنایی ندارید نگران نباشید، زیرا در ادامه به توضیح هر یک از آن‌ها خواهیم پرداخت. برای اینکه بهتر متوجه چیستی زبان جاوا اسکریپت شوید، در ابتدا باید جواب سوالاتی مانند زبان کامپایلری چیست و چه تفاوتی با زبان مفسری دارد؟، زبان برنامه نویسی سمت سرور و سمت کاربر به چه نوع زبان‌هایی گفته می‌شود؟ را بدانید. پس از درک این مفاهیم می‌توانید آموزش جاوا اسکریپت را شروع کنید.

همانطور که می‌دانید کامپیوترها تنها به زبان صفر و یک (Binary) صحبت می‌کنند و زبان دیگری را متوجه نمی‌شوند. ما در ابتدا برای برقراری ارتباط با ماشین‌ها سعی کردیم به زبان خود آنها، یعنی زبانی که به زبان صفر و یک نزدیک‌تر است، صحبت کنیم. به این نوع زبان‌ها که به صورت مستقیم با پردازنده در ارتباط‌اند، در اصطلاح، زبان‌های سطح پایین (Low Level) گفته می‌شود. از جمله این زبان‌ها می‌توان به اسمبلی اشاره کرد.

اما یادگیری و تسلط به این زبان‌ها برای برنامه نویسان فوق العاده سخت بود. بنابراین متخصصین تصمیم به ساخت زبان‌هایی گرفتند که به زبان انسان‌ها نزدیک‌تر باشد. در اصطلاح به این زبان‌ها، زبان‌های سطح بالا (High Level) می‌گویند. زبان‎های سطح بالایی مانند JavaScript کار را برای برنامه‎نویسان ساده‌تر کردند، زیرا ساختار نوشتاری و منطق آن‌ها بسیار به زبان انسان‌ها نزدیک‌تر شده است. پس می‌توان اینگونه نتیجه گرفت که آموزش جاوا اسکریپت نسبت به سایر زبان‌های برنامه نویسی سطح پایین ساده‌تر است.

همانطور که دیدید در تعریف زبان برنامه نویسی جاوا اسکریپت به این نکته اشاره شد که این زبان از نوع زبان‌های مفسری است. برای درک ماهیت زبان‌های برنامه نویسی مفسری ابتدا فکر کنید که شما یک مترجم هستید. برای ترجمه یک متن، دو راه بیشتر ندارید. یا باید آنچه را دریافت می‌کنید به صورت خط به خط و همزمان ترجمه کنید، یا کل مطلب را یک جا ترجمه کنید. این دقیقا همان تفاوت میان زبان‌های مفسری (Interpreter) و زبان‌های کامپایلری (Compiled) است.

\subsection{تاریخچه زبان جاوا اسکریپت}
جاوا اسکریپت اولین بار در می‌1995 در 10 روز توسط برندن ایچ، یکی از کارکنان شرکت Netscape متولد شد! در ابتدا این شرکت به این نتیجه رسیده بود که به صفحات وب پویا و جذاب‌تری احتیاج دارد. این اولین قدم به سوی ساخت زبانی ساده بود. آقای براندان ایچ از طرف این شرکت مامور شد که زبانی اسکریپتی برای صفحات وب و دست بردن در کدهای HTML بسازد. ماموریت آقای ایچ این بود زبانی را ارائه کند که نه تنها متخصصان برنامه نویسی از آن استقبال کنند، بلکه به راحتی مورد استفاده طراحان هم باشد.

این شرکت در ابتدا به فکر ارتقا و ساده سازی زبان Schema افتاد اما در نهایت به این نتیجه رسید که به زبانی شبیه جاوا اما با سینتکس ساده‌تر احتیاج دارد. در ابتدای کار اسم این زبان برنامه نویسی Mocha بود که بعد به Mona تغییر پیدا کرد. در سپتامبر همان سال اسم این زبان به LiveScript تغییر کرد و در آخر سریال تغییر اسم با انتخاب اسم JavaScript به اتمام  رسید.

نهایی شدن این اسم تنها به این دلیل بود که در آن روز‌ها زبان برنامه نویسی Java بسیار پرطرفدار شده بود. انتخاب این نام برای این زبان بسیار هوشمندانه بود. زیرا در آن زمان  این زبان با انتخاب این نام، توانست سهم زیادی از بازار جاوا را به خود اختصاص دهد. به هر حال در سال 1996 جاوا اسکریپت برای استاندارد شدن به سازمان ECMA سپرده شد. در نهایت اولین استاندارد جاوا اسکریپت با نام ECMAScript در سال 1997 منتشر شد. اولین اکما اسکریپت ECMA-262 و آخرین ورژن آن با اسم ECMAScript 2017 در ژوئن 2017 منتشر شد.

\subsection{نقاط قوت زبان جاوا اسکریپت}
هر یک از زبان هایی که در دنیای برنامه نویسی مورد استفاده قرار می‌گیرند نقاط قوت و ضعف هایی دارند که زبان جاوا اسکریپت هم از این موضوع مستثنا نیست. این زبان به دلیل مزایای فراوانی که دارد در میان برنامه نویسان از محبوبیت زیادی برخوردار است که به طور خلاصه به برخی از آنها اشاره می‌کنیم :
\begin{itemize}
	\item 
بر اساس بررسی سایت stackoverflow محبوب‌ترین زبان برنامه نویسی سال 2018 است
\item 
برای پردازش و اجرا به کامپایلر احتیاجی ندارد.
\item 
یادگیری جاوا اسکریپت نسبت به خیلی از زبان‌های برنامه نویسی راحت‌تر است.
\item 
به صورت کراس پلتفورم روی مرورگر‌ها یا پلتفرم‌های مختلف اجرا می‌شود.
\item 
نسبت به زبان‌های برنامه نویسی دیگر سبک‌تر و سریع‌تر است.
فریم ورک ها،کتابخانه‌ها و به صورت کلی ابزارهای بسیار زیادی را در اختیارتان قرار می‌دهد.
\item 
زبان بومی مرورگر وب است و در مرورگر کاربران پردازش می‌شود.
امکان ایجاد صفحات وب تعاملی و پویا را به برنامه نویسان می‌دهد.
\item 
در جواب عمل کاربران، عکس العمل نشان می‌دهد.

\end{itemize}
\subsection{نقاط ضعف زبان جاوا اسکریپت چیست؟}
 برخی از ضعف‌های این زبان برنامه نویسی عبارتند از :
\begin{itemize}
	\item 
دشواری در تشخیص دلیل خطا دادن و مشکل در دیباگ کردن
\item 
محدودیت در اجرای اسکریپت‌های جاوا اسکریپت با ایجاد محدودیت هایی جهت حفظ امنیت
\item 
اجرا نشدن بر روی مرورگرهای قدیمی
\item 
نفوذپذیری نسبت به اکسپلویت‌ها و عوامل مخرب
\item 
می تواند برای اجرای کدهای مخرب در کامپیوتر کاربران استفاده شود.
\item 
با رندر شدن متفاوت بر روی ابزارهای مختلف می‌تواند باعث ایجاد تناقض و نداشتن یکپارچگی شود.
\end{itemize}

\subsection{کاربرد جاوا اسکریپت}
پیش‌تر به محبوبیت زبان جاوا اسکریپت اشاره کردیم. این محبوبیت بی دلیل نیست چرا که با این زبانِ شی گرا شما قادر خواهید بود تا سایت‌های بی‌روح خود را جان بخشی کنید و با کاربران خود تعامل داشته باشید. یعنی می‌توانید فایل‌های انیمیشنی، صوتی و تصویری را روی سایت خود به نمایش بگذارید. همچنین می‌توانید روی سایت‌تان تایمر قرار دهید، رنگ‌ها را با حرکت موس تغییر دهید و بسیاری کارهای دیگر که باعث جذابیت بیشتر صفحات وب می‌شوند.

اما این تمام چیزی نیست که جاوا اسکریپت در اختیار شما قرار می‌دهد. شما با استفاده از این زبان می‌توانید شروع به ساخت برنامه‌های وب و موبایل و دسکتاپ کنید. برای این منظور می‌توانید از فریم‌ورک‌های مختلف JavaScript که مجموعه‌ای از کتابخانه‌ها را در اختیار شما قرار می‌دهند استفاده کنید. یکی از کارهای سرگرم کننده دیگری که می‌توانید از طریق این زبان انجام دهید، توسعه بازی‌های رایانه ای تحت مرورگر است.  پس به صورت کلی می‌توان کاربردهای زبان جاوا اسکریپت را به صورت زیر بیان کرد :
\begin{itemize}
	\item 
برنامه نویسی فرانت اند
\item 
برنامه نویسی بک اند با جاوا اسکریپت
\item 
برنامه نویسی نرم افزارهای موبایل
\item 
برنامه نویسی نرم افزارهای دسکتاپ

\end{itemize}



\section{NodeJS}

\subsection{Node.js چیست ؟}
Node.js یک پلتفرم سمت سرور مبتنی بر موتور جاوا اسکریپت گوگل کروم (V8 Engine) می‌باشد. Node.JS تمام چیزهایی که برای اجرای یک برنامه نوشته شده به زبان جاوا اسکریپت را نیاز دارید برایتان فراهم می‌کند. آقای Ryan Dahl در سال 2009 Node.JS را معرفی کرد تا نشان دهد جاوا اسکریپت قدرتمند‌تر از این حرف‌ها است که فقط برای پویاسازی صفحات وب در فرانت اند استفاده شود. در واقع به کمک Node.js زبان برنامه نویسی جاوا اسکریپ به جای اجرا درمرورگر در محیط سرور اجرا می‌شود. Node.js به شما اجازه می‌دهد به آسانی و سادگی برنامه‌های تحت شبکه مقیاس پذیر و بزرگ بنویسید.

جاوا اسکریپت از سال 1995 در حال پیشرفت بود. هر چند این زبان تا مدت‌ها قبل حضور موفقی در سمت سرور نداشت و و تلاش هایی که توسط برنامه نویسان انجام شده بود، به مرور زمان از ذهن توسعه دهندگان دیگر محو می‌شد. تا اینکه با معرفی نود جی اس در سال 2009 مهره برگشت و به مرور زمان جاوا اسکریپت بیشتر و بیشتر در سمت سرور مورد استفاده قرار گرفت.

\subsection{چرا باید از Node.js استفاده کنیم؟}
\subsubsection{Node.js بازدهی و انعطاف بالایی دارد}
نود در کنار V8 engine از زبان برنامه نویسی C++ استفاده کرده و سرعت بسیار بالایی دارد. هم V8 هم Node.js به صورت مرتب آپدیت شده و با قابلیت‌های جدید جاوا اسکریپت هماهنگ می‌شوند، همینطور بازدهی آنها بالاتر رفته و مشکلات امنیتی آنها نیز برطرف می‌شود. همینطور به دلیل استفاده از زبان جاوا اسکریپت انتقال فایل JSON (متداول‌ترین قالب انتقال داده در وب) به طور پیش فرض بسیار سریع خواهد بود.

\subsubsection{Node.js کراس پلتفرم است}
پلتفرم هایی مثل Electron.js یا NW.js به شما اجازه می‌دهند با نود جی اس برنامه‌های دسکتاپ بسازید. به این ترتیب می‌توانید برخی از کدهای برنامه تحت وب خود را در محیط ویندوز، لینوکس و مک اواس استفاده کنید. در واقع به کمک نود جی اس، همان تیمی که روی نسخه وب محصول کار می‌کنند، بدون نیاز به دانش تخصصی در زبان‌های C\# یا Objective C یا سایر زبان هایی که برای ساخت برنامه‌های Native به کار می‌روند، می‌توانند یک برنامه دسکتاپ بسازند.

\subsubsection{Node.js می‌تواند با میکروسرویس‌ها ترکیب شود}
اکثر پروژه‌های بزرگ در اول کار ساده بودند و در یک نسخه MVP معرفی شده بودند. اما به مرور زمان این سرویس‌ها بزرگتر شده و نیاز به اضافه کردن قابلیت‌های جدید در آنها حس می‌شد. گاهی وقت‌ها بزرگ شدن سرویس و اضافه کردن امکانات جدید به محصول می‌تواند برای تیم توسعه دهندگان تبدیل به یک کابوس شود. اما یک راه حل مناسب برای حل این مشکل استفاده از میکروسرویس است. میکروسرویس کمک می‌کند برنامه خود را بخش‌های کوچک تقسیم کنید که هر بخش می‌تواند توسط تیم متفاوت و حتی زبانی متفاوت نوشته شود. نود جی اس در کار با میکروسرویس‌ها عملکرد بسیار خوبی دارد.


\subsection{Node.js چه کاربردهایی دارد؟}
\subsubsection{ساخت برنامه‌های تک صفحه ای (SPA)}
SPA مخفف single-page app بوده و برنامه هایی گفته می‌شود که تمام بخش‌های آن در یک صفحه پیاده سازی می‌شود. از SPA بیشتر برای ساخت شبکه‌های اجتماعی، سرویس‌های ایمیل، سایت‌های اشتراک ویدئو و غیره استفاده می‌شود. یکی از معروف‌ترین سایت هایی که به این شکل ساخته شده است، سرویس اشتراک ویدئو یوتیوب است. از آنجایی که نود جی اس از برنامه نویسی نامتقارن یا asynchronous به خوبی پشتیبانی می‌کند، برای ساخت برنامه‌های SPA انتخاب خوبی به حساب می‌آید

\subsubsection{ساخت برنامه‌های RTA}
RTA مخفف real-time app می‌باشد. یعنی برنامه هایی که به صورت لحظه ای دارای تغییرات مختلفی هستند. به احتمال زیاد قبلا با این نوع برنامه‌ها کار کرده اید. برای مثال Google Sheets، Spreadsheets یا Slack از این دست برنامه‌ها هستند. در کل برنامه‌های تعاملی، ابزارهای مدیریت پروژه، کنفرانس‌های ویدئویی و صوتی و سایر برنامه‌های RTA عملیات‌های سنگین ورودی/خروجی انجام می‌دهند.

\subsubsection{ساخت بازی‌های آنلاین تحت مرورگر وب}
ایده ساخت چت روم جذاب است، اما جذابیت آن زمانی بیشتر می‌شود که یک بازی هم برای مرورگر وب بنویسید و کنار آن بازی یک چت روم هم ارائه کنید. به کمک نود جی اس می‌توان به توسعه بازی تحت وب پرداخت. در واقع با ترکیب تکنولوژی‌های HTML5 و ابزارهای جاوا اسکریپت ( مثل Express.js یا Socket.io یا غیره ) می‌توانید بازی‌های دوبعدی جذابی مثل Ancient Beast یا PaintWar بسازید.

\section{NoSQL}
\subsection{NoSQL چیست؟}
در برنامه نویسی سنتی، پایگاه‌های داده معمولا از نوع SQL هستند؛ که یک پایگاه داده رابطه ای یا Relational است. پایگاه‌های داده رابطه ای ساده هستند و کار کردن با آن‌ها معمولا بی دردسر و راحت است. اما این نوع از پایگاه‌های داده یک مشکل بزرگ دارند. این مشکل زمانی خود را نشان داد که غول‌های نرم افزاری دنیا مثل گوگل، آمازون و فیسبوک احتیاج به تحلیلِ داده‌های با حجم و تعداد بالا یا همان Big Data پیدا کردند.

پایگاه‌های داده رابطه ای به دلیل نوع ساختار خود، برای تحلیل داده‌های بزرگ غیر بهینه، ناکارا و همینطور کند بودند. البته در بعضی موارد هم استفاده از ساختار جدولی که در پایگاه‌های داده رابطه ای استفاده می‌شود تقریبا ناممکن بود. به همین دلیل ذخیره سازی حجم زیادی از داده‌های بی ساختار (Non-structured Data) سرعت و کارایی این پایگاه‌های داده را به شدت کاهش می‌داد. تا اینکه پایگاه‌های داده NoSQL پا به عرصه گذاشتند. پس همانطور که حدس می‌زنید، هدف اصلی ایجاد پایگاه‌های داده NoSQL کار با داده‌های بی ساختار و حجیم است.

گفتیم که مشکل پایگاه‌های داده مبتنی بر SQL از نوع ساختار آن‌ها ناشی می‌شود. اما این ساختار چگونه است و چرا چنین مشکلاتی را ایجاد می‌کند؟ برای فهمیدن پاسخ این سوال احتیاج داریم کمی با ساختار پایگاه‌های داده SQL آشنا شویم.

\subsection{پایگاه‌های داده NoSQL}
پایگاه‌های داده (Not Only SQL) NoSQL  برعکس نوع SQL از ساختارهای Schema غیر ثابت یا Dynamic Schema استفاده می‌کنند. این باعث می‌شود که برنامه نویسان احتیاجی به تشکیل ساختارهای سخت گیرانه مشخص، پیش از ایجاد پایگاه‌های داده را نداشته باشند. این پایگاه‌های داده می‌توانند انواع مختلفی داشته باشند و برعکس SQL برای ذخیره سازی داده‌ها از XML یا JSON استفاده می‌کنند. در ادامه انواع مختلفی از پایگاه‌های داده NoSQL را به شما معرفی می‌کنیم:
\begin{itemize}
	\item 
پایگاه‌های داده کلید-مقدار یا Key-Value Database: در این نوع از پایگاه داده اطلاعات در قالب جفت‌های کلید-مقدار یا Key-Value ذخیره می‌شود. کلیدها نقش شناسه هر داده را بازی می‌کند. یعنی می‌توانیم با استفاده از آن‌ها مقادیر مختلف داده را ذخیره یا پیدا کنیم. پایگاه‌های داده کلید-مقدار به دلیل ساده بودن در کارکرد، پرکاربرد‌ترین نوع پایگاه‌های داده NoSQL هستند.
\item 
پایگاه‌های داده ستونی یا Wide-Column Database: شاید تصور کنید پایگاه‌های داده ستونی همان پایگاه‌های داده رابطه ای هستند. اما این فقط ظاهر این گونه پایگاه‌های داده است که شبیه به نوع رابطه ای است. گفتیم که در پایگاه‌های داده رابطه ای لازم است که تعداد و نوع ویژگی‌های هر موجودیت و مقادیر داخل آن مشخص و ثابت باشد. این در حالی است که در پایگاه‌های داده ستونی، هر ستون در رکوردهای مختلف می‌تواند شامل داده هایی با ساختار و نوع متفاوت باشد.
\item 
پایگاه‌های داده سندی یا Document Database: در این گونه پایگاه‌های داده برای ذخیره سازی داده‌ها از اسناد JSON یا XML استفاده می‌کنیم. پایگاه‌های داده سندی معمولا برای ذخیره سازی و استفاده از داده‌های پراکنده و بی ساختار استفاده می‌شوند.
\item 
پایگاه‌های داده گرافی یا Graph Database: در این نوع از پایگاه‌های داده برای ذخیره سازی موجودیت‌ها و روابط بین آن‌ها از گراف استفاده می‌کنیم. پایگاه‌های داده گرافی برای مواردی که در آن‌ها به ایجاد ارتباط‌های متعدد بین جداول احتیاج داریم بسیار مناسب هستند.
\item 
پایگاه‌های داده چند مدله یا Multimodel Database: پایگاه‌های داده چند مدله ترکیبی از انواع دیگر پایگاه داده هستند. در این نوع پایگاه‌های داده می‌توانیم داده‌ها را به روش‌های مختلفی ذخیره، و از آن‌ها استفاده کنیم.

\end{itemize}
\subsection{مزیت‌های استفاده از NoSQL}
پایگاه‌های داده NoSQL مزیت‌های بسیار زیادی دارند که آن‌ها را برای سیستم‌های بزرگ و توزیع شده تبدیل به بهترین گزینه می‌کند. به طور کلی می‌توان این مزیت‌ها را به این شکل خلاصه کرد:
\begin{itemize}
	\item 
مقیاس پذیری بالا (Scalability): پایگاه‌های داده NoSQL می‌توانند به راحتی با روش مقیاس پذیری افقی یا Horizontal Scaling گسترش پیدا کنند. این ویژگی باعث کم شدن پیچیدگی و هزینه مقیاس دادن به نرم افزار یا Scale کردن آن می‌شود.
\item 
کارایی بالا (Performance): در سیستم‌های توزیع شده NoSQL با تکثیر خودکار داده‌های NoSQL در سرورهای متعدد در سراسر دنیا، تاخیر در ارسال پاسخ از طرف سرور به پایین‌ترین حد ممکن می‌رسد.
\item 
دسترسی بالا (Availability): در سیستم‌های توزیع شده NoSQL به دلیل کپی شدن خودکار داده‌ها در سرورهای مختلف، با از دسترس خارج شدن یک یا چند سرور، پایگاه داده همچنان قابل دسترس و پاسخگو است.
\end{itemize}

\section{MongoDB}
\subsection{mongo db چیست؟}
مونگو دیبی (Mongo DB) یکی از معروف‌ترین پایگاه داده‌های No SQL است که ساختار منعطفی دارد و بیشتر در پروژه هایی با حجم بالای داده استفاده می‌شود. این پایگاه داده پلتفرمی متن باز و رایگان است و با مدل داده‌های مستند گرا (Document - Oriented) کار می‌کند و در ویندوز، مکینتاش و لینوکس قابل استفاده است. مقادیر داده ای ذخیره شده در مونگو دیبی، با دو کلید اولیه (Primary Key) و ثانویه (Secondary Key) مورد استفاده قرار می‌گیرند.

مونگو دیبی شامل مجموعه ای از مقادیر است. این مقادیر به صورت سندهایی (Document) هستند که با اندازه‌های مختلف، انواع مختلفی از داده‌ها را در خود جای داده اند. این مسئله باعث شده که مونگو دیبی بتواند داده هایی با ساختار پیچیده مانند داده‌های سلسله مراتبی و یا آرایه ای را در خود ذخیره کند.

\subsection{ویژگی‌های mongo db}
\begin{itemize}
	\item 
مونگو دیبی به علت مستند گرا بودن مدل ذخیره داده‌ها در مقایسه با دیتابیس‌های رابطه ای بسیار منعطف‌تر و مقیاس پذیر‌تر است و بسیاری از نیازمندی‌های کسب و کارها را برطرف می‌کند.
\item 
این پایگاه داده برای تقسیم داده‌ها و مدیریت بهتر سیستم از شاردینگ (Sharding) استفاده می‌کند. شاردینگ به معنی تکه تکه کردن است و در لود بالای شبکه انجام می‌شود. به گونه ای که دیتابیس به چند زیربخش تقسیم می‌شود تا روند پاسخ دهی به درخواست هایی که از سمت سرور می‌آید، راحت‌تر شود.
\item 
داده‌ها با دو کلید اولیه و ثانویه قابل دسترسی هستند و هر فیلدی قابلیت کلید شدن را دارد. این امر زمان دسترسی و پردازش داده را بسیار سریع می‌کند.
\item 
همانند سازی (Replication ) یکی دیگر از خصوصیات مهم مونگو دیبی است. در این تکنیک از یک داده به عنوان داده اصلی کپی هایی تهیه شده و بخش‌های دیگری از سیستم پایگاه داده ذخیره می‌شود. در صورت از بین رفتن و یا مخدوش شدن این داده، داده‌های کپی شده به عنوان داده اصلی و جایگزین مورد استفاده قرار می‌گیرند.
\end{itemize}

\subsection{روش کار mongo db}
در دیتابیس‌های رابطه ای داده‌ها به شکل رکورد (Record) نگهداری می‌شوند اما در مونگو دیبی، ساختار نگهداری داده‌ها به شکل سند است. هر سند از نوع Binary JSON یا BSON است و دارای فیلدهای کلید و مقدار می‌باشد.

برای اجرا کردن کدهایی که در مونگو دیبی نوشته شده است باید از طریق Mongo Shell اقدام کرد. مونگو شل رابط تعاملی دیتابیس و برنامه نویس محسوب می‌شود و به آن‌ها اجازه ارسال کوئری (Query) و به روزرسانی داده‌ها را می‌دهد.

\subsection{مزایا و معایب mongo db}
دیتابیس‌های رابطه ای دارای اسکیما (Schema) هستند. یعنی ساختار خاصی برای داده هادر نظر گرفته و مدل‌های محدودی را ذخیره می‌کنند. اما مونگو دیبی و به طور کلی دیتابیس‌های NoSQL در برابر پذیرش داده هایی با توع مختلف بسیار منعطف هستند و این مزیت مهمی برای برنامه نویسان محسوب می‌شود. مقیاس پذیری این پایگاه داده باعث استفاده از آن در پروژه هایی می‌شود که با کلان داده‌ها (Big Data) سروکار دارند.

علاوه بر مزایای گفته شده مشکلاتی نیز در مونگو دیبی وجود دارد که ممکن است دردسرساز شود. این دیتابیس در استفاده از کلید خارجی (Foreign Key) برای داده‌ها ضعف دارد و ممکن است پایداری داده‌ها و یکپارچگی سیستم را به هم بریزد. همچنین در خوشه بندی داده‌های موجود در این پایگاه داده، تنها می‌توان یک گره (Node) را به عنوان گره اصلی (Master) انتخاب کرد که اگر از بین برود، ممکن است مرتب سازی زیرگره‌های آن از بین برود. این مشکل در پایگاه داده کاساندرا (Cassandra) برطرف شده است.
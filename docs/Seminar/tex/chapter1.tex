% !TeX root = ../Report.tex

%کلیات طرح ، اهداف، انگیزه

\addcontentsline{toc}{section}{مقدمه}
\section*{مقدمه}
امروزه با پیشرفت تکنولوژی و افزایش ضریب استفاده از اینترنت، شکل و مفهوم کار دستخوش تغییرات زیادی شده است. این تغییرات فرصت‌های فراوانی را در اختیار کارآفرینان، متخصصین و افراد علاقه‌مند به تکنولوژی‌های جدید می‌گذارد.
فریلنسینگ یا دورکاری، شیوه‌ای از کار است که در آن متخصصین، بدون نیاز به حضور دائمی در دفتر کار یا تعهد بلندمدت به یک شرکت، به‌صورت پروژه‌ای، خدمات حرفه‌ای خود را به کسب‌وکارها ارائه می‌دهند. همگام شدن با تغییرات و شیوه‌های جدید، رمز موفقیت هر فرد یا سازمان موفقی است. به‌منظور تسهیل ارائه خدمات و افزایش رقابت‌پذیری و همچنین فراهم آوردن فرصت کاری برابر برای متخصصین، اقدام به طراحی پلتفرم برون‌سپاری پروژه‌های کوچک و بزرگ نموده است.
 متخصصین فریلنسر یا دورکار، با مهارت‌هایی مانند برنامه‌نویسی، طراحی سایت، طراحی گرافیک و دیزاین، تولید محتوا، ترجمه، سئو و سایر مهارت‌ها قادرند بدون محدودیت‌های معمول مکانی یا زمانی، با ارائه خدمات خود به صورت آنلاین و اینترنتی کسب درآمد نمایند. همچنین از اهداف ما فراهم آوردن امکان کسب درآمد از طریق دورکاری برای معلولین و افراد کم‌توان بوده است. کارآفرینان و صاحبان کسب‌وکار با امکان دسترسی به بانک اطلاعاتی بزرگی از متخصصین از تمامی نقاط می‌توانند خدمات حرفه‌ای موردنیاز خود را باکیفیت بهتر و قیمت مناسب‌تری تحویل بگیرند و بهره‌وری و رقابت‌پذیری خود را افزایش دهند. این خدمات می‌تواند شامل طراحی وب‌سایت، برنامه‌نویسی، ساخت اپلیکیشن موبایل، طراحی لوگو و کاتالوگ، تهیه محتوای نوشتاری و تصویری، ساخت تیزر تبلیغاتی، تایپ، ترجمه متون و سایر خدمات حرفه‌ای باشد.
هدف اصلی فراهم آوردن امکان دسترسی به بازار کار و اشتغال برای متخصصین در گوشه و کنار کشور حتی در شهرها و روستاهای کوچک و دورافتاده است. ما معتقدیم مهارت و دانش در همه‌جا وجود دارد و برابر بودن فرصت‌های کاری برای افراد باعث رونق و شکوفایی بیشتر اقتصادی می‌شود. معتقدیم فعالیت ما ضمن کمک به افزایش اشتغال‌زایی و ایجاد فرصت‌های کاری برابر، بستری تعاملی و دسترسی سریع و حرفه‌ای برای کسب‌وکارها، صاحبان ایده و کارآفرینان برای رفع نیازهای حرفه‌ای و ارتقا کسب‌وکار ایجاد می‌کند.
\section{تعریف و بیان کار}
 به‌عنوان یک سایت فریلنسینگ امکان کسب درآمد از طریق دورکاری رو برای فریلنسرها فراهم کرده. همچنین برون‌سپاری پروژه‌های طراحی وب و برنامه‌نویسی، تولید محتوا، طراحی گرافیک، تایپ، ترجمه و بسیاری مهارت‌های دیگر هم از این طریق امکان‌پذیر است.

 در این پروژه موارد زیر از دانشجو درخواست می‌شد:
\begin{enumerate}
	\item
سیستم توانایی تعریف و پرداخت هزینه انجام پروژه را دارا باشد.
\item
این سیستم دارای انواع گزارش ‌گیری از فریلنسرها و کارفرماها باشد.
\item
سیستم توانایی تعامل کارفرما و فریلنسر را دارا باشد.
\end{enumerate}

\section{سابقه و ضرورت}
سابقه فریلنسری به جک نیلز برمی‌گردد. jack Nilles برای اولین بار ایده ارتباط از راه دور را مطرح کرد.
در آن روزها یعنی دهه 1970 مشکل حمل‌ونقل در شهرها وجود داشت و کاملاً آزاردهنده بود که افراد ساکن مناطق دور برای کارهای روزمره خود به شهرهای اصلی سفر کنند.
حتی دولت نگران این موضوع بود و به دنبال راه‌حلی برای آن. همیشه گفته می‌شود که نیاز مادر اختراع است.
با اختراع تلفن توسط گراهام بل در سال 1876 تمام دنیا شگفت‌زده شد. در دهه 1970 تلفن برای کار در دفاتر ایالات‌متحده امریکا مورداستفاده قرار گرفت و این همان چیزی بود که jack Nilles آن را مورد هدف قرارداد و به این فکر افتاد که از تلفن برای حل مشکل استفاده کند و درنهایت مفهوم کار در منزل با تلفن را معرفی کرد.
آن زمان اینترنت هنوز دنیا را شگفت‌زده نکرده بود و بسیاری از این ایده کار کردن دلگیر شدند.
نگرانی‌های زیادی در مورد کار در منزل وجود داشت و باید ثابت می‌شد که این نوع کار کردن مؤثر است.
بنابراین او در آزمایشگاه دانشگاه کالیفرنیا شروع به انجام آزمایشاتی کرد و توانست نتایج خوبی را کسب کند.
این‌گونه بود که وی توانست کمک‌های مالی خود را از موسسه ملی علوم دریافت کند و حتی دولت هم از این ایده حمایت کرد، چون‌که تنها راه‌حل برای حل مشکل حمل‌ونقل بود و می‌توانست مصرف سوخت را کاهش دهد و میزان افزایش آلودگی در آمریکا را کاهش دهد.
Frank Schiff یکی دیگر از افرادی بود که از کار در خانه حمایت کرد. وی عمدتا نگران افزایش مصرف بنزین و سایر سوخت ها بود. او به پیشنهادات jack Nilles کمک کرد و به عنوان رئیس کمیته توسعه اقتصادی از ایده کار از راه دور دفاع کرد. او در مقاله ای با عنوان “کار در خانه می تواند بنزین را نجات دهد” به توصیفِ زیبایی مزایای کار از راه دور پرداخت. وی بیشتر این کار را با نام ” محل کار انعطاف پذیر” مطرح می کرد. حتی در آن روز ها وی توانست مفهوم استفاده از اینترنت را به صورت بسیار تحریک آمیزی مطرح کند. در نتیجه این مقاله در ایالات متحده محبوبیت زیادی به دست آورد و ایده فریلنسری موجب شد که همه به بررسی مزایا و معایب فریلنسری بپردازند. Gil Gordon نام دیگری است که در اینجا باید به آن اشاره کنیم، وی کمک زیادی به انتشار ارتباط از راه دور کرد و در کنار جک نیلز دیدگاه های علمی در مورد مزایای دورکاری یا فریلنسری مطرح کرد.
با گذشت زمان استفاده از اینترنت روز به روز بیشتر رواج یافت. در اواخر قرن بیست و بیست و یکم افراد فریلنسر برای انجام کاری خاص با یکدیگر ارتباط برقرار می کنند و وقتی که کار انجام می گرفت اتصالات منحل می شد و بدین ترتیب آنها دوباره آزاد می شدند. از آنجا که این افراد از طریق اینترنت به یکدیگر وصل می شدند اصلاح جدید e-lancers متولد شد. در سال های اولیه قرن بیست و یکم، کل حوزه اینترنتی با سرعت غیر قابل تصوری رونق پیدا کرد و اکنون تقریبا همه به اینترنت دسترسی دارند. اولین بار در سال ۱۹۹۹ و امروز در حدود بیش از ۱۰۰۰ پرتابل فریلنسری در اینترنت تاسیس شده است. با افزایش تعداد چنین سایت ها و شرکت هایی رقابت در بازار به صورت تصاعدی در حال افزایش است و این برای مشتری و فریلنسر یک مزیت محسوب می شود. فریلنسرها می توانند از این فرصت استفاده کنند تا بازارهایی که دارای درآمد خوبی هستند را پیدا کنند و مشتریان هم می توانند به دنبال پلت فرم هایی باشند که کمترین هزینه را برای آنها دارند.
مشتری ها به طور کلی به فریلنسرهای با تجربه اعتماد دارند و ممکن است پروژه های مختلف خود را به آنها بسپارند. بنابراین بسیار مهم است که در زمینه های خاص تخصص کافی را بدست آورید. بعضی از مشتریان حاضرند برای یک کار با کیفیت بالا هزینه های خوبی پرداخت کنند. بنابراین برای اینکه این اتفاق بیافتد، برخورد خوب یکی از موضوعات ضروری است که باید در نظر داشته باشید. یکی دیگر از عواملی که می تواند تا حد زیادی روی درآمد یک فریلنسر تاثیر بگذارد زمینه ای است که وی برای کار انتخاب می کند. برای مثال یک برنامه نویس نرم افزار درآمد بسیار بیشتری نسبت به یک محتوا نویس کسب می کند. دلیل این امر هم این است که تعداد محتوانویسان بسیار زیاد است و تعداد برنامه نویسان نرم افزاری کم. عرضه کمتر باعث افزایش تقاضا و در نتیجه افزایش قیمت ها می شود. امیدواریم این تاریخچه فریلنسری مختصر نظر شما را به خود جلب کرده باشد.
\begin{itemize}
	\item
	 سایت‌های فریلنسری به کارفرمایان این امکان را می‌دهند که از میان هزاران فرد متخصص در یک حوزه خاص بتوانند پروژه خود را با قیمتی مشخص به فرد مورد نظر تحویل دهند
	 \item
	کارفرما نیاز ندارد تا برای انجام یک یا چند پروژه، کارمند استخدام کند و می‌تواند با برونسپاری پروژه، کار را به فریلنسرها بسپارد
	\item
	به دلیل محیط رقابتی‌ای که در سایت‌های فریلنسری وجود دارد، کارفرمایان می‌توانند با قیمتی مناسب‌تر پروژه‌های خود را تحویل دهند. (البته در صورت کاهش بیش‌از حد قیمت افراد مجرب به انجام پروژه شما ترغیب نخواهند شد)
	\item
	کارفرمایان می‌توانند از میان فریلنسرها، کارمندان مناسبی را انتخاب کنند و آنها را از مزایای آن شغل مطلع کنند.
\end{itemize}

\section{هدف‌ها}
به راحتی بتوان برای پروژه‌های خود، نیروی متخصص پیدا کرد.
 پلتفرم فریلنسینگ و دورکاری اینترنتی، جهت استخدام فریلنسرهای حرفه‌ای در ایران است. از جمله پروژه‌هایی که می‌توانید آن‌ها را برون سپاری کنید و یا انجام دهید، می‌توان به: برنامه نویسی، ساخت اپلیکیشن و نرم افزار، طراحی سایت، تولید محتوای سایت و شبکه‌های اجتماعی، تایپ و ترجمه، عکاسی، طراحی (لوگو، بنر، پوستر، چهره و...)، نگارش مقاله و پروپوزال، بازاریابی، دیجیتال مارکتینگ، سئو SEO (بهینه‌سازی موتورهای جست و جو)، ادیت و ویرایش عکس، ساخت و تدوین کلیپ و تیزر، ساخت اینفوگرافیک، مدل‌سازی، ویرایش فایل‌های صوتی، آهنگسازی، پروژه‌های مهندسی، علوم تجربی، هنر، معماری، طراحی داخلی و... اشاره کرد.

\section{کاربردها}
فریلنسر یا آزادکار، شخصیست که با توجه به مهارت‌ها و تخصصی که دارد، به صورت اینترنتی و دورکاری، پروژه میگیرد و انجام می‌دهد و برای کارفرما می‌فرستد. فریلنسرها از طریق پروژه های دورکاری کسب درآمد می‌کنند. به این شیوه کاری فریلنسینگ گفته می شود. فریلنسرها، میتوانند از هر جایی که مایلند، حتی در خانه و هر ساعتی که دوست دارند، کار کنند و پروژه انجام دهند. فریلنسر ها ساعت کاری و محل کار ثابتی ندارند و خودشان محل و ساعات کاریشون رو انتخاب میکنند. فریلنسینگ نوعی سبک زندگی و کار است که امکان دورکاری و کسب درآمد از منزل رو برای فریلنسرها فراهم میکنه. بنابر بررسی‌ها تا سال 2025، نصف جمعیت شاغلان دنیا رو فریلنسرها تشکیل میدن. همچنین برونسپاری پروژه های طراحی وب و برنامه نویسی، تولید محتوا، طراحی گرافیک، تایپ، ترجمه و بسیاری مهارت های دیگه هم از این طریق امکان پذیر است.
\section{مراحل کار}

\section{روش کار}

\section{ساختار گزارش}
در این گزارش به پنج فصل زیر پرداخته شده است:
\begin{itemize}
	\item
	در فصل اول به کلیات و اهداف پرداخته شده است.
	\item
	در فصل دوم به مفاهیم و دانش فنی پرداخته شده است.
	\item
	در فصل سوم به تحلیل نرم‌افزار پرداخته شده است.
	\item
	در فصل چهارم به طراحی نرم‌افزار پرداخته شده است.
	\item
	در فصل پنجم به جمع‌بندی و پیشنهادات پرداخته شده است.
\end{itemize}

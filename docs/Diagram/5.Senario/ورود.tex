\documentclass[20pt,a5paper]{report}
\usepackage{graphicx,geometry}
\usepackage{multirow,hhline}
\usepackage[hidelinks]{hyperref}

\usepackage{xepersian}

\settextfont{Adobe Arabic}
\geometry{a5paper,top = 0.5cm, left = 1cm, right = 1cm, bottom = 0.5cm}


\begin{document}
\noindent \textbf{مورد استفاده:}
ورود
\\
\textbf{شرح مختصر :UC}
در این قسمت بازدیدکننده به سایت وارد میشود و به عنوان کاربر شناخنه میشود
\\
\textbf{پيش شرط:}
ثبت نام در سایت.
\\
\textbf{سناريو اصلی:}
\begin{enumerate}
\item 
شروع
\item 
بازدید کننده دکمه ورود را انتخاب میکند و سیستم فرم ورود را به بازدید کننده نمایش میدهد.
\item 
بازدید کننده فرم ورود را تکمیل میکند و با دکمه ارسال، فرم تکمیل شده را به سیستم ارسال میکند.
\item 
سیستم فرم ورود را بررسی میکند و اطلاعات را از بانک اطلاعات دریافت میکند .
\item 
کاربر در سایت شناسایی و وارد میشود٫
\item 
پایان
\end{enumerate}
\textbf{پس شرط:}
ندارد .
\\
\textbf{سناريوهای فرعی:}
\\ \\
\textbf{سناريو فرعی 1:}
خطا در اطلاعات فرم ورود
\\
\textbf{شرح مختصر :UC}
این سناریو در مرحله ۴ سناریو اصلی در صورت خطا در اطلاعات فرم ورود اجرا میشود.
\begin{enumerate}
\item 
شروع
\item 
اطلاعات فرم بررسی میشود و خطاها مشخص میشوند.
\item 
یک پیغام به بازدیدکننده نمایش داده میشود و درخواست اصلاح اطلاعات فرم را دارد.
\item 
از مرحله 3 سناریو اصلی ادامه پیدا میکند.
\item 
پایان
\end{enumerate}
\textbf{سناريو فرعی 2:}
کاربر با موفقیت وارد میشود.
\\
\textbf{شرح مختصر :UC}
این سناریو در مرحله ۴ سناریو اصلی در صورت موفقیت آمیز بودن ورود اجرا میشود.
\begin{enumerate}
\item 
شروع
\item 
اطلاعات فرم بررسی میشود و یک پیغام به کاربر نمایش داده میشود که ورود موفقیت آمیزی داشته.
\item 
از مرحله 5 سناریو اصلی ادامه پیدا میکند.
\item 
پایان
\end{enumerate}

\textbf{پس شرط:}
ندارد .


\centering
\vfill
\lr{\LaTeX}
\end{document}
